\subsection{Sardinas-Patterson}

\par L'algorithme de \textit{Sardinas-Patterson} calcul en temps polynomial si un ensemble de mots de code est sans préfixe. Si tel est le cas un mot de code produit sera \textbf{uniquement décodable}.




\subsubsection{Fonctionnement}
\par L'algorithme est décrit inductivement par cette définition : \\
 $ S_1 = C^{-1}C  \setminus \epsilon$ \\
$S_{i+1} = C^{-1}S_i\cup S_i^{-1}C$
\par Ici $\epsilon$ représente le mot vide. On utilise la notion de \textbf{quotients résiduels} c'est à dire pour $N^{-1}D = \left\{ y | xy \in D, x \in N \right\} $. En d'autre terme, l'ensemble résultant est l'ensemble des suffixes des mots dans \textbf{D} tel que leurs préfixes sont dans \textbf{N}.

\subsubsection{Algorithme}
\begin{minted}{c++}

\end{minted}

\subsubsection{Exemple :}
\par Prenons l'exemple, \textit{\{1, 011, 01110, 1110, 10011\}} : \\ 

\begin{tabular}{|c|c|}
    \hline
    Indice & S \\
    \hline
    0 & 110, 0011, 10\\
    \hline
    1 & 10, 0, 011\\
    \hline
\end{tabular}

\par On remarque la présence de \textbf{011} qui est un mot appartenant au code. Ce code n'est donc pas à décodage unique, donc pas sans préfixe. 

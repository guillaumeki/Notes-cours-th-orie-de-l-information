\par L'algorithme MTF (pour move-to-front : "déplacer vers l'avant") est un système de transformation de flot utilisé notamment dans le domaine de la compression de données en informatique. 

\subsubsection{Fonctionnement}
\par Cet algorithme consiste à remplacer chaque caractère par un indice donné par un tableau évoluant de manière dynamique. 
Le tableau est tout d'abord initialisé en rangeant les caractères utilisé pour le codage comme ceci : \\

\begin{tabular}{|c|c|c|c|c|c|c|c|c|c|c|c|c|c|c|c|}
    \hline
    Indice & 0 & 1 & 2 & 3 & 4 & ... & 25\\
    \hline
    Caractère & A & B & C & D & E & ... & Z\\
    \hline
\end{tabular}

\par Lorsqu'un caractère est lu, son indice est émis, puis ce caractère est placé en première position et tous les autres caractères décalés.

\subsubsection{Algorithme}
\begin{minted}{c++}
/*@function MTF.
*@param string chaine, la chaine à encoder.
*@param map<char,int> Alphabet, l'alphabet.
*@return vector<size_t> indices, la chaine encodé dans une suite d'indices.
*/
vector<size_t> MTF(string chaine, map<char, int> Alphabet){
	
    vector<size_t> indices;
    map<char,int>::iterator it;
    size_t k(0);

    for(size_t i(0); i < chaine.length(); ++i)
    {
	
	    k = Alphabet.at(chaine[i]);
	    indices.push_back(k);
	    Alphabet[chaine[i]] = 0;
	    for(it = Alphabet.begin(); it != Alphabet.find(chaine[i]) && it != Alphabet.end(); ++it){
		    it->second++;
		}
			
    }
	
    return indices;
	
}
\end{minted}


\subsubsection{Spécificité}
\par Lorsque des caractères semblables se suivent, le flux émis contiendra beaucoup de 0 alors que dans une compression statistique (type codage de Huffman) augmentera considérablement le Taux de compression de données.

\subsubsection{Exemple :}
Exemple de l'application MTF à yabadabadoo\$ :

\begin{tabular}{|c|c|c|}
    \hline
    Caractère lu & Indices & Alphabet \\
    \hline
      &  & \$abcdefghijklmnopqrstuvwxyz \\
      \hline
    y & 25 & y\$abcdefghijklmnopqrstuvwxz\\
    \hline
     ya & 25;2 & ay\$bcdefghijklmnopqrstuvwxz\\
    \hline
     yab & 25;2;3 & bay\$cdefghijklmnopqrstuvwxz\\
    \hline
     yaba & 25;2;3;1 & aby\$cdefghijklmnopqrstuvwxz\\
    \hline
     yabad & 25;2;3;1;5 & daby\$cefghijklmnopqrstuvwxz \\
    \hline
     yabada & 25;2;3;1;5;1 & adby\$cefghijklmnopqrstuvwxz \\
    \hline
     yabadab & 25;2;3;1;5;1;2 & bady\$cefghijklmnopqrstuvwxz\\
    \hline
     yabadaba & 25;2;3;1;5;1;2;1 & abdy\$cefghijklmnopqrstuvwxz \\
    \hline
     yabadabad & 25;2;3;1;5;1;2;1;2 & daby\$cefghijklmnopqrstuvwxz \\
    \hline
     yabadabado & 25;2;3;1;5;1;2;1;2;16 & odaby\$cefghijklmnpqrstuvwxz \\
    \hline
     yabadabadoo & 25;2;3;1;5;1;2;1;2;16;0 & odaby\$cefghijklmnpqrstuvwxz \\
    \hline
     yabadabadoo\$ & 25;2;3;1;5;1;2;1;2;16;0;5 & \$odabycefghijklmnpqrstuvwxz \\
    \hline
\end{tabular}